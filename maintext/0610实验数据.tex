6月10日回传xps数据一份:
\begin{figure}[H]
    \begin{subfigure}[b]{0.48\textwidth}
        \includegraphics[width=\textwidth]{1/C1s.png}
        \caption{nZr=0.78/PvP=1.725,800℃,C1s}
        \label{nZr=0.78/PvP=1.725,800℃,C1s}
    \end{subfigure}
    \hfill
    \begin{subfigure}[b]{0.48\textwidth}
        \includegraphics[width=\textwidth]{1/O1s.png}
        \caption{nZr=0.78/PvP=1.725,800℃,O1s}
        \label{nZr=0.78/PvP=1.725,800℃。O1s}
    \end{subfigure}
    \hfill
    \begin{subfigure}[b]{0.48\textwidth}
        \includegraphics[width=\textwidth]{1/Si2p.png}
        \caption{nZr=0.78/PvP=1.725,800℃,Si2p}
        \label{nZr=0.78/PvP=1.725,800℃,Si2p}
    \end{subfigure}
    \hfill
    \begin{subfigure}[b]{0.48\textwidth}
        \includegraphics[width=\textwidth]{1/Zr3d.png}
        \caption{nZr=0.78/PvP=1.725,800℃,Zr3d}
        \label{nZr=0.78/PvP=1.725,800℃,Zr3d}
    \end{subfigure}
    \caption{nZr=0.78,800℃组}
\end{figure}

\begin{figure}[H]
    \begin{subfigure}[b]{0.48\textwidth}
        \includegraphics[width=\textwidth]{2/C1s.png}
        \caption{nZr=0.78/PvP=1.725,900℃,C1s}
        \label{nZr=0.78/PvP=1.725,900℃,C1s}
    \end{subfigure}
    \hfill
    \begin{subfigure}[b]{0.48\textwidth}
        \includegraphics[width=\textwidth]{2/O1s.png}
        \caption{nZr=0.78/PvP=1.725,900℃,O1s}
        \label{nZr=0.78/PvP=1.725,900℃,O1s}
    \end{subfigure}
    \hfill
    \begin{subfigure}[b]{0.48\textwidth}
        \includegraphics[width=\textwidth]{2/Si2p.png}
        \caption{nZr=0.78/PvP=1.725,900℃,Si2p}
        \label{nZr=0.78/PvP=1.725,900℃,Si2p}
    \end{subfigure}
    \hfill
    \begin{subfigure}[b]{0.48\textwidth}
        \includegraphics[width=\textwidth]{2/Zr3d.png}
        \caption{nZr=0.78/PvP=1.725,900℃,Zr3d}
        \label{nZr=0.78/PvP=1.725,900℃,Zr3d}
    \end{subfigure}
    \caption{nZr=0.78,900℃组}
\end{figure}

\begin{figure}[H]
    \begin{subfigure}[b]{0.48\textwidth}
        \includegraphics[width=\textwidth]{3/C1s.png}
        \caption{nZr=0.64/PvP=1.725,1000℃,C1s}
        \label{nZr=0.64/PvP=1.725,1000℃,C1s}
    \end{subfigure}
    \hfill
    \begin{subfigure}[b]{0.48\textwidth}
        \includegraphics[width=\textwidth]{3/O1s.png}
        \caption{nZr=0.64/PvP=1.725,1000℃,O1s}
        \label{nZr=0.64/PvP=1.725,1000℃,O1s}
    \end{subfigure}
    \hfill
    \begin{subfigure}[b]{0.48\textwidth}
        \includegraphics[width=\textwidth]{3/Si2p.png}
        \caption{nZr=0.64/PvP=1.725,1000℃,Si2p}
        \label{nZr=0.64/PvP=1.725,1000℃,Si2p}
    \end{subfigure}
    \hfill
    \begin{subfigure}[b]{0.48\textwidth}
        \includegraphics[width=\textwidth]{3/Zr3d.png}
        \caption{nZr=0.64/PvP=1.725,1000℃,Zr3d}
        \label{nZr=0.64/PvP=1.725,1000℃,Zr3d}
    \end{subfigure}
    \caption{nZr=0.64,1000℃组}
\end{figure}

\cref{nZr=0.78/PvP=1.725,800℃,Si2p}与\cref{nZr=0.78/PvP=1.725,900℃,Si2p}对比发现\cref{nZr=0.78/PvP=1.725,900℃,Si2p}并没有四个子峰,这是比较奇怪的一个事情,与此同时,
\cref{nZr=0.64/PvP=1.725,1000℃,Si2p}的子峰之间的结合能区别也不显著e;