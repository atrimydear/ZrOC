\section{0401效果}
\begin{figure}{H}
    \begin{subfigure}[b]{0.48\textwidth}
        \includegraphics[width=\textwidth]{2-30401.jpg}
        \caption{nZr:PvP=0.64:1.725;喷射速率$1.8ml/h$}
        \label{fig:2030401a}
    \end{subfigure}
    \hfill
    \begin{subfigure}[b]{0.48\textwidth}
        \includegraphics[width=\textwidth]{2-110401.jpg}
        \caption{nZr:PvP=0.64:1.725;喷射速率$1.5ml/h$}
        \label{fig:2110401b}
    \end{subfigure}
    
    \begin{subfigure}[b]{0.48\textwidth}
        \includegraphics[width=\textwidth]{1-20401.jpg}
        \caption{nZr:PvP=0.66:2.75;喷射速率$1.5ml/h$}
        \label{fig:1020401a}
    \end{subfigure}
    \hfill
    \begin{subfigure}[b]{0.48\textwidth}
        \includegraphics[width=\textwidth]{1-70401.jpg}
        \caption{nZr:PvP=0.66:2.75;喷射速率$1.8ml/h$}
        \label{fig:1070401b}
    \end{subfigure}
    \caption{三月三十一日四份样品sem图}
    \label{三月三十一日四份样品sem图}
\end{figure}

将\Cref{fig:1020401a}与\Cref{2g}对比, 我们可以发现, \Cref{fig:1020401a}的纤维相对不光滑, 对高温下力学性能来说可能会有影响。形成原因来看,不太可能是因为挤出速率更慢与PvP:Zr的含量配比,根据同行的经验来看,
有可能是因为封箱的问题:封箱了箱子里湿度更低,失去了水汽的电场均匀化作用,导致胶液在飞行过程中精细拉伸较弱,因而形成的纤维束不太均匀。需要实现湿度的可控化。

    \begin{figure}[H]
        \begin{subfigure}[b]{0.48\textwidth}
        \includegraphics[width=\textwidth]{1-2纤维直径高斯分布统计0401.png}
        \caption{\Cref{fig:1020401a}纤维直径分布}
        \label{tab:1070401}
        \end{subfigure}
        \hfill
        \begin{subfigure}[b]{0.48\textwidth}
        \includegraphics[width=\textwidth]{1-7纤维直径高斯分布统计0401.png}
        \caption{\Cref{fig:1070401b}纤维直径分布}
        \label{tab:1070401}
        \end{subfigure}
        \hfill

        \begin{subfigure}[b]{0.48\textwidth}
        \includegraphics[width=\textwidth]{2-3纤维直径高斯分布统计0401.png}
        \caption{\Cref{fig:2030401a}纤维直径分布}
        \label{tab:2030401}
        \end{subfigure}
        \hfill
        \begin{subfigure}[b]{0.48\textwidth}
        \includegraphics[width=\textwidth]{2-11纤维直径高斯分布统计0401.png}
        \caption{\Cref{fig:2110401b}纤维直径分布}
        \label{tab:2110401}
        \end{subfigure}
        \caption{\Cref{三月三十一日四份样品sem图}纤维直径分布图}
        \label{0401纤维直径分布}
    \end{figure}

    \begin{figure}[h]
        \includegraphics[width=\textwidth]{0401样品RL系数.png}
        \caption{四月一号样品RL系数}
        \label{tab:四月一号样品RL系数}
    \end{figure}

        可以看到,1、2、4的样品的RL非常不错,有继续研究的价值,当然导师希望我们先复现十多份二号样先好准备后续研究。