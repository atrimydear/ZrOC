\documentclass{ctexart}
\usepackage{graphicx}
    \graphicspath{{picture/}}
\usepackage{gbt7714}
\usepackage{subcaption}
% 加入 amsmath 和 booktabs,确保 cleveref 在 amsmath 之后加载
\usepackage{amsmath}
\usepackage{booktabs}
\usepackage{tabularx}
\usepackage{cleveref}
    \crefname{table}{表}{表}
    \crefname{figure}{图}{图}
\usepackage{float}
\usepackage[version=4]{mhchem}
\bibliographystyle{gbt7714-numerical}

\begin{document}
    \section{实验目标}\cite{WGCL202201014}
        通过二分法不断迭代寻找合适的参数
        \subsection{目标参数}
            电压、喷射距离、滚筒转速、挤出速率、滑台移速、nZr:PVP配比

            需要注意的是,电压和喷射距离本质上是服务于同一个参数: 胶液飞行速度。这两个参数应当整体调节。
            
            \paragraph{预想机理}通过控制胶液飞行速率避免胶液在飞行途中聚团,从而减少附着纤维中的聚团现象;挤出速率:避免在飞行速
            率过慢时挤出较多导致胶液束过粗;滑台移动速率:减少滑台变速抖动,抑制胶液偏聚;配比:控制纤维的分布;
            滚筒转速:尽快分离前后附着胶液的空间分布。
    \section{设备改良}
        铝箔纸换硅油纸、提高箱体气密性
    \section{实验计划}
        \begin{tabular}{|ccc|}
            \hline
            次序&参数&最初预设调参节点\\\hline
            1&nZr: PVP&1.5、2.5\\\hline
            2&滚筒转速&200+n40rpm\\\hline
            3&滑台移速&10mm/h\\\hline
        \end{tabular}
        
        电压与喷射距离(喷射速率)的调节需要进行数学计算
    \section{0321改良效果}

    \begin{figure}[H]
        \begin{subfigure}[b]{0.45\textwidth}
            \includegraphics[width=\textwidth]{2-4.png}
            \caption{标准参数:修正为nZr:PVP=2:0.6}
            \label{2g}
        \end{subfigure}
        \hfill
        \begin{subfigure}[b]{0.45\textwidth}
            \includegraphics[width=\textwidth]{3-3.png}
            \caption{标准参数}
            \label{标准参数}
        \end{subfigure}
    \end{figure}

    \Cref{2g}可以发现此时纤维均匀度大幅提高,但纤维束与纤维束之间的直径差距开始显现明显,目测在十纳米和数百纳米之间波动,纤维束之间较之\Cref{标准参数}更疏松。
    \Cref{标准参数}可以发现此时纤维不均匀,单条纤维束上直径差距大,在1微米到3微米之间波动,而纤维束之间出现了粘连成块状的区域,多条纤维束之间存在并列粘连,较之\Cref{2g}更密集,密度更高,预测密度较之\Cref{2g}
    更低。
    \section{0401效果}
\begin{figure}{H}
    \begin{subfigure}[b]{0.48\textwidth}
        \includegraphics[width=\textwidth]{2-30401.jpg}
        \caption{nZr:PvP=0.64:1.725;喷射速率$1.8ml/h$}
        \label{fig:2030401a}
    \end{subfigure}
    \hfill
    \begin{subfigure}[b]{0.48\textwidth}
        \includegraphics[width=\textwidth]{2-110401.jpg}
        \caption{nZr:PvP=0.64:1.725;喷射速率$1.5ml/h$}
        \label{fig:2110401b}
    \end{subfigure}
    
    \begin{subfigure}[b]{0.48\textwidth}
        \includegraphics[width=\textwidth]{1-20401.jpg}
        \caption{nZr:PvP=0.66:2.75;喷射速率$1.5ml/h$}
        \label{fig:1020401a}
    \end{subfigure}
    \hfill
    \begin{subfigure}[b]{0.48\textwidth}
        \includegraphics[width=\textwidth]{1-70401.jpg}
        \caption{nZr:PvP=0.66:2.75;喷射速率$1.8ml/h$}
        \label{fig:1070401b}
    \end{subfigure}
    \caption{三月三十一日四份样品sem图}
    \label{三月三十一日四份样品sem图}
\end{figure}

将\Cref{fig:1020401a}与\Cref{2g}对比, 我们可以发现, \Cref{fig:1020401a}的纤维相对不光滑, 对高温下力学性能来说可能会有影响。形成原因来看,不太可能是因为挤出速率更慢与PvP:Zr的含量配比,根据同行的经验来看,
有可能是因为封箱的问题:封箱了箱子里湿度更低,失去了水汽的电场均匀化作用,导致胶液在飞行过程中精细拉伸较弱,因而形成的纤维束不太均匀。需要实现湿度的可控化。

    \begin{figure}[H]
        \begin{subfigure}[b]{0.48\textwidth}
        \includegraphics[width=\textwidth]{1-2纤维直径高斯分布统计0401.png}
        \caption{\Cref{fig:1020401a}纤维直径分布}
        \label{tab:1070401}
        \end{subfigure}
        \hfill
        \begin{subfigure}[b]{0.48\textwidth}
        \includegraphics[width=\textwidth]{1-7纤维直径高斯分布统计0401.png}
        \caption{\Cref{fig:1070401b}纤维直径分布}
        \label{tab:1070401}
        \end{subfigure}
        \hfill

        \begin{subfigure}[b]{0.48\textwidth}
        \includegraphics[width=\textwidth]{2-3纤维直径高斯分布统计0401.png}
        \caption{\Cref{fig:2030401a}纤维直径分布}
        \label{tab:2030401}
        \end{subfigure}
        \hfill
        \begin{subfigure}[b]{0.48\textwidth}
        \includegraphics[width=\textwidth]{2-11纤维直径高斯分布统计0401.png}
        \caption{\Cref{fig:2110401b}纤维直径分布}
        \label{tab:2110401}
        \end{subfigure}
        \caption{\Cref{三月三十一日四份样品sem图}纤维直径分布图}
        \label{0401纤维直径分布}
    \end{figure}

    \begin{figure}[h]
        \includegraphics[width=\textwidth]{0401样品RL系数.png}
        \caption{四月一号样品RL系数}
        \label{tab:四月一号样品RL系数}
    \end{figure}

        可以看到,1、2、4的样品的RL非常不错,有继续研究的价值,当然导师希望我们先复现十多份二号样先好准备后续研究。
    \begin{figure}[H]
    \includegraphics[width=\textwidth]{folder1.png}
    \caption{从上至下nZr=0.64/0.78/0.64}
\end{figure}
    6月10日回传xps数据一份:
\begin{figure}[H]
    \begin{subfigure}[b]{0.48\textwidth}
        \includegraphics[width=\textwidth]{1/C1s.png}
        \caption{nZr=0.78/PvP=1.725,800℃,C1s}
        \label{nZr=0.78/PvP=1.725,800℃,C1s}
    \end{subfigure}
    \hfill
    \begin{subfigure}[b]{0.48\textwidth}
        \includegraphics[width=\textwidth]{1/O1s.png}
        \caption{nZr=0.78/PvP=1.725,800℃,O1s}
        \label{nZr=0.78/PvP=1.725,800℃。O1s}
    \end{subfigure}
    \hfill
    \begin{subfigure}[b]{0.48\textwidth}
        \includegraphics[width=\textwidth]{1/Si2p.png}
        \caption{nZr=0.78/PvP=1.725,800℃,Si2p}
        \label{nZr=0.78/PvP=1.725,800℃,Si2p}
    \end{subfigure}
    \hfill
    \begin{subfigure}[b]{0.48\textwidth}
        \includegraphics[width=\textwidth]{1/Zr3d.png}
        \caption{nZr=0.78/PvP=1.725,800℃,Zr3d}
        \label{nZr=0.78/PvP=1.725,800℃,Zr3d}
    \end{subfigure}
    \caption{nZr=0.78,800℃组}
\end{figure}

\begin{figure}[H]
    \begin{subfigure}[b]{0.48\textwidth}
        \includegraphics[width=\textwidth]{2/C1s.png}
        \caption{nZr=0.78/PvP=1.725,900℃,C1s}
        \label{nZr=0.78/PvP=1.725,900℃,C1s}
    \end{subfigure}
    \hfill
    \begin{subfigure}[b]{0.48\textwidth}
        \includegraphics[width=\textwidth]{2/O1s.png}
        \caption{nZr=0.78/PvP=1.725,900℃,O1s}
        \label{nZr=0.78/PvP=1.725,900℃,O1s}
    \end{subfigure}
    \hfill
    \begin{subfigure}[b]{0.48\textwidth}
        \includegraphics[width=\textwidth]{2/Si2p.png}
        \caption{nZr=0.78/PvP=1.725,900℃,Si2p}
        \label{nZr=0.78/PvP=1.725,900℃,Si2p}
    \end{subfigure}
    \hfill
    \begin{subfigure}[b]{0.48\textwidth}
        \includegraphics[width=\textwidth]{2/Zr3d.png}
        \caption{nZr=0.78/PvP=1.725,900℃,Zr3d}
        \label{nZr=0.78/PvP=1.725,900℃,Zr3d}
    \end{subfigure}
    \caption{nZr=0.78,900℃组}
\end{figure}

\begin{figure}[H]
    \begin{subfigure}[b]{0.48\textwidth}
        \includegraphics[width=\textwidth]{3/C1s.png}
        \caption{nZr=0.64/PvP=1.725,1000℃,C1s}
        \label{nZr=0.64/PvP=1.725,1000℃,C1s}
    \end{subfigure}
    \hfill
    \begin{subfigure}[b]{0.48\textwidth}
        \includegraphics[width=\textwidth]{3/O1s.png}
        \caption{nZr=0.64/PvP=1.725,1000℃,O1s}
        \label{nZr=0.64/PvP=1.725,1000℃,O1s}
    \end{subfigure}
    \hfill
    \begin{subfigure}[b]{0.48\textwidth}
        \includegraphics[width=\textwidth]{3/Si2p.png}
        \caption{nZr=0.64/PvP=1.725,1000℃,Si2p}
        \label{nZr=0.64/PvP=1.725,1000℃,Si2p}
    \end{subfigure}
    \hfill
    \begin{subfigure}[b]{0.48\textwidth}
        \includegraphics[width=\textwidth]{3/Zr3d.png}
        \caption{nZr=0.64/PvP=1.725,1000℃,Zr3d}
        \label{nZr=0.64/PvP=1.725,1000℃,Zr3d}
    \end{subfigure}
    \caption{nZr=0.64,1000℃组}
\end{figure}

\cref{nZr=0.78/PvP=1.725,800℃,Si2p}与\cref{nZr=0.78/PvP=1.725,900℃,Si2p}对比发现\cref{nZr=0.78/PvP=1.725,900℃,Si2p}并没有四个子峰,这是比较奇怪的一个事情,与此同时,
\cref{nZr=0.64/PvP=1.725,1000℃,Si2p}的子峰之间的结合能区别也不显著e;
    \section{综述表}
        % Scale the whole table to \textwidth so each row stays on one line
\noindent{\scriptsize
\resizebox{\textwidth}{!}{%
\begin{tabular}{@{}lcccc@{}}
\toprule
Material & Tickness (nm) & Minimum RL(dB) & EAB(GHz) & Ref. \\
\midrule
Fe/SiC/PCS&2.25&-46.3&4.6(<-20dB)&\cite{hou2017electrospinning}\\
MXene&2.7/2.1&-54.1(2.7)&7.76(2.1)&\cite{zhang2023electrospun}\\
Fe/Co@C-CNFs&1.08/1.22&-18.66(1.08)&4.2(1.22)&\cite{RN57}\\
PAN/\ce{Fe3O4}&4&-11.3&7&\cite{https://doi.org/10.1002/app.38027}\\
TiN/carbon-paraffin composites&1.9&-41.8&3.9&\cite{WEI20181488}\\
SiC&1.9&-57.8&12&\cite{WEI20181488}\\
\ce{ZnO/Co}&3.0/2.6&-68.4(3.0)&5.9(2.6)&\cite{doi:10.1021/acsanm.8b01303}\\
\ce{P-CNF/Fe}&4.1&-44.86&3.28&\cite{ZUO20194474}\\
\ce{CeO2/NC}&2.5&-42.95&8.48&\cite{RN63}\\
\ce{Zr/SiC}&3.5&-48.6&3.2&\cite{ZHANG2022167036}\\
\ce{ZrC}&1.25/1.0&-25.77(1.25)&3.04(1.0)&\cite{GUO2023235}\\
\ce{ZrO2/ZrC/ZrB2}&2.4&-21&8.64&\cite{WANG2024136442}\\
\ce{ZrO2/ZrB2/C}&4.0&-54&3.1&\cite{YANG2025100988}\\
ZS-HNFA-x&2.4&-53.2&6.4&\cite{ZHU2025164079}\\
\ce{ZrO2/CF-rGO}&3.14&-62.99&8.19&\cite{YIN2025112628}\\
\ce{ZrO2/C}&2.5&-36&7.3&\cite{WANG2024104818}\\
\bottomrule
\end{tabular}%
}% end resizebox
} % end scriptsize
\bibliography{referrence}

\end{document}